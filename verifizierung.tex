\newpage
\chapter{Verifizierung}
Beim Modellieren und Simulieren kann nach Roy \cite{roy} prinzipiell in zwei Fehlerquellen unterschieden werden. Es gibt physikalische Modelierungsfehler und mathematische Fehler. Ersteres wird der Sparte der Validierung zugeordnet, die sich unter anderem mit falschen Vereinfachungen beschäftigt oder dem Gültikeitsbereich bestimmter Modelle. Dies steht bei der Prüfung eines implementierten Verfahrens allerdings an letzter Stelle, da dafür die Richtigkeit der Gleichungen und des Codes vorrausgesetzt werden muss.
Die Verifizierung der Gleichungen sowie die Verifizierung des Codes sind beides rein mathematische Verfahren, die lediglich prüfen ob die Gleichungen richtig gelöst werden. Die Validierung hingegen überprüft, ob die richtigen Gleichungen gelöst werden.  \cite{tremblay}.

\section{Str"omungsmechanische Grundlagen}
Um eine Str"omung physikalisch beschreiben zu k"onnen, werden mehrere Erhaltungss"atze angewendet:
\begin{itemize}
	\item Massenerhaltung
	\item Impulsverhaltung (in vektorieller Form)
	\item Energieerhaltung
\end{itemize}