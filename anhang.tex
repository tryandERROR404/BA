\newpage
\chapter*{Anhang A}
In diesem Abschnitt soll das Prinzip der Methode der Manufactured Solutions am Beispiel der zweidimensionalen Eulergleichung Gl. \ref{eq:euler} gezeigt werden. Daf"ur werden die von Glasby \cite{glasby2013comparison} vorgestellten Gleichungen als Manufactured Solutions eingesetzt:
\begin{equation}\label{key1}
	\rho = 0,5 (\sin(x^+y^)+1,5)
\end{equation}
\begin{equation}\label{key2}
	u = \sin(x^+y^2)+0,5
\end{equation}
\begin{equation}\label{key3}
	v = 0,1(\cos(x^+y^2)+0,5)
\end{equation}
\begin{equation}\label{key4}
	e = 0,5(\cos(x^2+y^2)+1,5)
\end{equation}
Wie Gl. \ref{zustandsvariablen} entnommen werden kann, braucht SUNWinT konservative Zustandsvaraiblen, weshalb die oben stehenden Gleichungen entsprechend miteinander multipliziert werden m"ussen.
Um nun die entsprechenden Quellterme berechnen zu k"onnen, m"ussen die entsprechenden konservativen Eulergleichungen gel"ost werden. Der "ubersichtlichkeit wegen werden Gl. \ref{eq:euler} hier nochmals anders formuliert:
\begin{equation}\label{eq:firstSource}
	\frac{\partial \rho}{\partial t}+\frac{\partial(\rho u)}{\partial x} + \frac{\partial (\rho v)}{\partial y} = f_{source1}
\end{equation}
\begin{equation}\label{eq:secondSource}
	\frac{\partial \rho u}{\partial t}+\frac{\partial(\rho u^+p)}{\partial x} + \frac{\partial (\rho u v)}{\partial y} = f_{source2}
\end{equation}
\begin{equation}\label{eq:thirdSource}
\frac{\partial \rho v}{\partial t} + \frac{\partial (\rho v u)}{\partial x} + \frac{\partial(\rho v^+p)}{\partial y} = f_{source3}
\end{equation}
\begin{equation}\label{eq:fourthSource}
\frac{\partial \rho e}{\partial t} + \frac{\partial ((\rho e+p)u)}{\partial x} +\frac{\partial ((\rho e+p)v)}{\partial y} = f_{source4}
\end{equation}

In diese Gleichungen k!nne nun unsere manufactured solutions eingesetzt werden um die Quellterme $f_{source1}$-$f_{source4}$ zu bestimmen. Da Gl. \ref{key1} - \ref{key4} nicht von der Zeit abh"angen verschwindet jeweils der erste Term, die darauf folgeden werden nach den "ublichen mathematischen Ableitungsregeln abgeleitet. Da die Ableitungen je nach Ausgangsgleichung und gew"ahlten Manufactured solutions teilweise sehr un"ubersichtilich werden, empfiehlt es sich diese mit einer Software wie Matlab zu erzeugen. 