\newpage
\chapter*{Anhang A}
In diesem Abschnitt soll das Prinzip der Methode der Manufactured Solutions am Beispiel der zweidimensionalen Eulergleichung Gl. \ref{eq:euler} gezeigt werden. Dafür werden die von Glasby \cite{glasby2013comparison} vorgestellten Gleichungen als Manufactured Solutions eingesetzt:
\begin{equation}\label{key1}
	\rho = 0,5 (\sin(x^+y^)+1,5)
\end{equation}
\begin{equation}\label{key2}
	u = \sin(x^+y^2)+0,5
\end{equation}
\begin{equation}\label{key3}
	v = 0,1(\cos(x^+y^2)+0,5)
\end{equation}
\begin{equation}\label{key4}
	e = 0,5(\cos(x^2+y^2)+1,5)
\end{equation}
Wie Gl. \ref{zustandsvariablen} entnommen werden kann, braucht SUNWinT konservative Zustandsvaraiblen, weshalb die oben stehenden Gleichungen entsprechend miteinander multipliziert werden müssen.
Um nun die entsprechenden Quellterme berechnen zu können, müssen die entsprechenden konservativen Eulergleichungen gelöst werden. Der Übersichtlichkeit wegen werden Gl. \ref{eq:euler} hier nochmals anders formuliert:
\begin{equation}\label{eq:firstSource}
	\frac{\partial \rho}{\partial t}+\frac{\partial(\rho u)}{\partial x} + \frac{\partial (\rho v)}{\partial y} = f_{source1}
\end{equation}
\begin{equation}\label{eq:secondSource}
	\frac{\partial \rho u}{\partial t}+\frac{\partial(\rho u^+p)}{\partial x} + \frac{\partial (\rho u v)}{\partial y} = f_{source2}
\end{equation}
\begin{equation}\label{eq:thirdSource}
\frac{\partial \rho v}{\partial t} + \frac{\partial (\rho v u)}{\partial x} + \frac{\partial(\rho v^+p)}{\partial y} = f_{source3}
\end{equation}
\begin{equation}\label{eq:fourthSource}
\frac{\partial \rho e}{\partial t} + \frac{\partial ((\rho e+p)u)}{\partial x} +\frac{\partial ((\rho e+p)v)}{\partial y} = f_{source4}
\end{equation}

In diese Gleichungen könne nun unsere manufactured solutions eingesetzt werden um die Quellterme $f_{source1}$-$f_{source4}$ zu bestimmen. Da Gl. \ref{key1} - \ref{key4} nicht von der Zeit abhängen verschwindet jeweils der erste Term, die darauf folgeden werden nach den üblichen mathematischen Ableitungsregeln abgeleitet. Da die Ableitungen je nach Ausgangsgleichung und gewählten Manufactured solutions teilweise sehr unübersichtilich werden, empfiehlt es sich diese mit einer Software wie Matlab zu erzeugen. 