\newpage
\chapter{Discontinous Galerkin Verfahren}
In diesem Kapitel wird das in SUNWinT implementierte DG Verfahren skizziert. Da ein r"aumlich und zeitlich getrennter Ansatz gew"ahlt wurde, wird zun"achst auf die r"aumliche und im Anschluss daran auf die zeitliche Diskretisierung eingegangen.

\section{R"aumliche Diskretisierung}
Als Grundlage dient die Euler-Gleichung in differentieller konservativer Form
\begin{equation}\label{eq:ausgansgleichungDG}
	\underbrace{\frac{\partial \vec{U}}{\partial t}+ \frac{\partial}{\partial \vec{x}} \vec{F(U)}}_{\substack{\vec{R(U)}}} = 0
\end{equation}
Diese Gleichung ist f"ur die exakte L"osung gleich null, bei einer numerischen N"aherung jedoch wird R(U) nicht exakt null sein. Um allerdings m"oglichst nahe ran zu kommen, werden gibt es verschiedene Ans"atze.
Bei der Methode der gewichteten Residuen, die in SUNWinT implementiert ist, wird nun nicht das betrachtete Intervall eines Raumes diskretisiert, sondern der Funktionenraum. Dazu w"ahlt man eine beliebige Anzahl elementarer Basisfunktionen. Dies k"onnen beispielsweise Polynome, wie in SUNWinT verwendet\footnote{Basissfunktionen in SUNWinT: $1, x, x^2, x^3$, ...}, sein aber auch trigonometrische Funktionen o."a. w"are m"oglich. Die Ansatzfunktion ist nun eine Linearkombination aus diesen Basisfunktionen\footnote{Ansatzfunktion: $v = a_{0}1+a_{1}x+a_{2}x^2$...}. Die daf"ur ben"otigten Koefizienten $a_{i}$ werden auch Freiheitsgrade genannt und so bestimmt, dass die Ansatzfunktion eine m"oglichst gute Approximation darstellt. Das bedeutet die Freiheitsgrade werden so bestimmt, dass der lokale Fehler (die Abweichung von null) m"oglichst gering ist.
So wird nun U aus Gl. \ref{eq:ausgansgleichungDG} approximiert zu 
\begin{equation}\label{eq:Uansatz}
	\vec{\hat{U}}(x, t) = \sum_{n=0}^N U_{k}(t)b_{k}(\vec{x})
\end{equation}
Da ein zeitlich und r"aumlich getrennter Ansatz verwendet wird sind die Freiheitgrade $U_{k}$ nur von der Zeit t abh"angig und die Basisfunktionen $b_{k}$ nur von der Position im Raum. Die Anzahl der Basisfunktionen und Freiheitgrade ist von der Ordnung des Verfahrens abh"angig.
Nachdem die Basisfunktionen gew"ahlt wurden und in Gl. \ref{eq:ausgansgleichungDG} eingesetzt wurde, muss nun das Residuum m"oglichst verschwinden. Auch dazu gibt es mehrere verschiedene Ans"atze, wie beispielsweise die Methode der kleinsten Quadrate, die fordert, dass der Betrag des Residuums bez"uglich der Freiheitgrade m"oglichst verschwindet. Hier wird allerdings nur auf die Galerkin-Methode eingegangen, da diese im Code von SUNWinT verwendet wird.
Die Galerkin-Methode fordert, dass das Integral des Produkt der Basisfunktionen mit dem Residuum auf dem gew"ahlten Intervall verschwindet. Das hei"st, das Residuum muss orthogonal auf dem Raum der Ansatzfunktionen stehen. Dazu muss
\begin{equation}\label{eq:galerkinAnsatz}
	\int_{\Omega} v_{j}(x) R(\text{\^{U}(x,t)}) d\Omega  \overset{!}{=} 0
\end{equation}
gelten.
Entscheidend bei diesem Ansatz ist, dass $v_{j}(x)=b_{k}(x)$ für $j=k$ gilt, also die Basisfunktionen identisch mit den Gewichtungsfunktionen sind. Zu beachten ist au"serdem, dass U bzw. \^{U} der L"osungsvektor ist, der f"ur jede Zustandsvariable eine Zeile hat. Das in Gl. \ref{ref:galerkinAnsatz} beschriebe Gleichungssystem besitzt daher \textit{Anzahl der Basisfunktionen (bzw. Ordnung des Verfahrens) $\cdot$ Anzahl der Zustandsvariablen} Gleichungen.
Entscheidend f"ur das discontinous Galerkin Verfahren ist, dass die Ansatzfunktion \^{U} nicht "uber den gesamten Raum, sondern nur innerhalb einer Zelle stetig sein muss. Teilt man nun den Raum in beliebig viele Elemente E, die sich nicht "uberlappen d"urfen, und addiert anschließend alle Elemente gilt:
\begin{equation}\label{eq:galerkin}
	\sum (\int_{E}v_{j}\frac{\partial U}{\partial t}d\Omega+ \int_{E} \underbrace{v_{j}}_{\substack{g(x)}}\underbrace{\nabla F(U)}_{\substack{f'(x)}}{\Omega}) = 0
\end{equation}
Zur Vereinfachung der Rechenschritte wird wieder die Notation aus Gl \ref{eq:ausgansgleichungDG} verwendet. Außerdem wird im Folgenden immer lediglich ein Element betrachtet. Wie Gl. \ref{eq:galerkin} entnommen werden kann, bietet sich nun der hintere Term f"ur eine partielle Integration an, mit der man auch die aufw"andige partielle Ableitung von F({U}) umgeht.
\begin{equation}\label{eq:basisDG}
	\underbrace{\int_{E}v_{j}\frac{\partial U}{\partial t}d\Omega}_{\substack{I}} +	
	\underbrace{\oint_{\partial E}v_{j}F(U)\vec{n}d\sigma}_{\substack{II}}-
	\underbrace{\int_{E}\nabla v_{j}F(U)d\Omega}_{\substack{III}} = 0
\end{equation}

\begin{itemize}
	\item Teil I:\\
	Für U wird der diskretisierte Ansatz aus Gl. \ref{eq:Uansatz} eingesetzt. Wie bereits erwähnt gilt für den Galerkinansatz, dass $v_{j}=b_{k}$ für $j=k$. Außerdem ist nur $U_{k}$ von t abhängig, weshalb die Ableitung vor das Integral gezogen werden kann:
	\begin{equation}\label{eq:I}
		\int_{E} v_{j} \frac{\partial \sum_{n=0}^{N} U_{k}(t)b_ {k}(\vec{x})}{\partial t} d\Omega=	\frac{d}{d t}\int_{E} v_{j} v_{k} U_{k}(t) d\Omega
	\end{equation}
	Wobei 
	\begin{equation}\label{eq:massenmatrix}
		M = \int_{E} v_{j} v_{k} d\Omega
	\end{equation}
	als Massenmatrix dargestellt wird. Durch die Wahl von orthogonalen Basisfunktionen ist dies eine Diagonalmatrix mit $M_{ij}=0$ für $i \neq j$. Letztlich wird kann Gl. \ref{eq:I} auf
	\begin{equation}\label{key}
		M \cdot \frac{d \vec{u}}{d t}
	\end{equation}
	vereinfacht werden
	
	\item Teil II: Oberflächenintegral
	
	\item Teil III: Zellintegral
	Da $v_{k}$ bereits bekannt ist muss zunächst $F(U)$ eingesetzt werden und anschließend mit den Diskretisierungen von U ausgedrückt werden.
	\begin{equation}\label{eq:III}
		\int_{E}\nabla v_{j}F \left(\begin{array}{c} \rho \\ \rho u\\ \rho v \\ \rho w \\ \rho E\end{array}\right)d\Omega =
		\int_{E}\nabla v_{j}F \left(\begin{array}{c} \rho u + \rho v + \rho w \\ \rho u^2 + p + \rho u v + \rho u w \\ \rho v^2 + p + \rho v u + \rho v w \\
		 \rho w^2 + p + \rho w u + \rho w v \\ (\rho E+p)(u+v+w)\end{array}\right)d\Omega 
	\end{equation}
	Dabei gilt beispielsweise
	\begin{equation}\label{eq:AnsatzRhoU}
		\rho u = \sum_{n=0}^{N} b_{i} \rho u_{i} 
	\end{equation}
	\begin{equation}\label{eq:AnsatzRhoU^2p}
			\rho u^{2} + p = \frac{(\sum_{n=0}^{N} b_{i} \rho u_{i})^{2}}{\sum_{n=0}^{N} b_{i}\rho_{i}}+(\kappa-1)(\sum_{n=0}^{N} (b_{i} \rho E_{i}) - \frac{1}{2} \frac{(\sum_{n=0}^{N} b_{i} \rho u_{i})^{2}}{2})
	\end{equation}
\end{itemize}
