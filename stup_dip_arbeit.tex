%&latex
% headsepline: Linie am oberen Blattrand unterhalb der Seitennummer
% bibtotoc: Aufnahme des Literaturverzeichnisses ins Inhaltsverzeichnis
\documentclass[a4paper,headsepline,bibtotoc]{scrreprt}
\makeatletter
\DeclareOldFontCommand{\rm}{\normalfont\rmfamily}{\mathrm}
\DeclareOldFontCommand{\sf}{\normalfont\sffamily}{\mathsf}
\DeclareOldFontCommand{\tt}{\normalfont\ttfamily}{\mathtt}
\DeclareOldFontCommand{\bf}{\normalfont\bfseries}{\mathbf}
\DeclareOldFontCommand{\it}{\normalfont\itshape}{\mathit}
\DeclareOldFontCommand{\sl}{\normalfont\slshape}{\@nomath\sl}
\DeclareOldFontCommand{\sc}{\normalfont\scshape}{\@nomath\sc}
\makeatother
% Einstellungen bez. des 'scrreprt'-Stils
% Caption Schriftstil und -Groesse
\renewcommand{\capfont}{\footnotesize}
\renewcommand{\caplabelfont}{\footnotesize\bfseries}
\typearea{15}  %Einstellung des Verh�ltnisses Gr��e des Textes zur Papiergr��e
% Sprache
\usepackage[german]{babel}
\selectlanguage{german}
\setlength{\parindent}{0pt}

\addto\extrasgerman{\renewcommand{\figurename}{Abb.}}
\addto\extrasgerman{\renewcommand{\tablename}{Tab.}}

% Bilder
\usepackage[rflt]{floatflt}
\usepackage{epsfig,wrapfig}

% Programmablaufplaene (Struktogramme, Nassi-Schneidermann-Diagramme)
% Dieses Paket ist nicht standardm��ig im CIP-Pool installiert
% \usepackage{nassi}

% Mathematische Symbole
\usepackage{amsmath,amssymb}

% Tabellen
\usepackage{longtable,lscape}
\usepackage{multirow}
\usepackage{tabularx}

% Kopfzeilen
\usepackage{fancyheadings}
\pagestyle{plain}
\renewcommand{\chaptermark}[1]{\markboth{#1}{}}
\renewcommand{\sectionmark}[1]{\markboth{\thesection\ #1}{}}
\lhead[\fancyplain{}{\sl\leftmark}]%
      {\fancyplain{}{\sl\leftmark}}
\rhead[\fancyplain{}{\sl\thepage}]%
      {\fancyplain{}{\sl\thepage}}
\cfoot{}

% Aufgabenstellung
\usepackage{iagkopf}

% Listenerscheinung
\setlength{\itemsep}{0ex}
\setlength{\parsep}{0ex}
\setlength{\parskip}{2mm}


\begin{document}
\sloppy

% Seitennumerierung bis zum Beginn der Einleitung auf kleine roemische Zahlen setzen
\pagenumbering{roman}

% neue Befehle
%\newcommand{\file}[1]{{\sffamily\slshape #1}}
\newcommand{\file}[1]{\mdseries\textsl{\textsf{#1}}}
\newcommand{\sbr}[1]{\texttt{#1}}
\newcommand{\var}[1]{\mdseries\textsl{\texttt{#1}}}
\newcommand{\cmd}[1]{\uppercase{\texttt{#1}}}

% Titelseite
\title{Titel der Studien/Diplomarbeit}

\author{Studien/Diplomarbeit \\
        von \\
        cand.\ aer.\ ...........   }

\publishers{durchgef"uhrt am \\
            Institut f"ur Aerodynamik und Gasdynamik \\
            der Universit"at Stuttgart \\
	    und bei (Name der Firma / \\
	    Forschungseinrichtung etc. bei externen Arbeiten).\\[5ex]
            Stuttgart, im (Monat) (Jahr)}

\date{}

\maketitle

% Aufgabenstellung einbinden
\addcontentsline{toc}{chapter}{Aufgabenstellung}
\pagestyle{plain}
\newpage
\chapter*{Aufgabenstellung}
An diese Stelle wird die Aufgabenstellung der Arbeit eingebunden.


% �bersicht
\addcontentsline{toc}{chapter}{"Ubersicht}
\newpage
\chapter*{"Ubersicht}
Nach der Titelseite des Berichtes und dem Aufgabenblatt soll das Wesentliche aus dem Inhalt der Arbeit in
wenigen S"atzen zusammengefasst werden. Diese "Ubersicht soll keine Formeln und m"oglichst keine Literaturhinweise enthalten.


% Inhaltsverzeichnis
\addcontentsline{toc}{chapter}{Inhaltsverzeichnis}
\tableofcontents

\clearpage
% Nomenklatur
\addcontentsline{toc}{chapter}{Nomenklatur}
\newpage
\chapter*{Nomenklatur}
\begin{tabular}{lll}
  \vspace{1mm}
  $a_i$            & [-]            & Polynomkoeffizient \\
  \vspace{1mm}
  $c_f$            & [-]            & Reibungswiderstandsbeiwert der
turbulent umstr"omten ebenen Platte \\
  \vspace{1mm}
  $c_{w_V}$        & [-]            & volumenbezogener
Widerstandsbeiwert \\
  \vspace{1mm}
  $D$              & [m]             & Durchmesser \\
  \vspace{1mm}
  $L$              & [m]            & K"orperl"ange \\
  \vspace{1mm}
  $n_{krit.}$      & [-]            & kritischer Anfachungsfaktor
(relevant f"ur die Umschlagsberechnung) \\
  \vspace{1mm}
  $Re_L$           & [-]            & l"angenbezogene Reynoldszahl \\
  \vspace{1mm}
  $Re_V$           & [-]            & volumenbezogene Reynoldszahl \\
  \vspace{1mm}
  $r$              & [m]            & Radius \\
  \vspace{1mm}
  $U_{\infty}$     & [$m/s$]        & Anstr"omgeschwindigkeit \\
  \vspace{1mm}
  $V$              & [$m^3$]        & Volumen \\     
   \vspace{1mm}
  $W$              & [$N$]          & Widerstand \\      
  \vspace{1mm}
  $x, y, z$        & [m]            & kartesische Koordinaten \\


  \vspace{1mm}
  $\alpha$        & [$^{\circ}$] & Anstellwinkel \\
  \vspace{1mm}
  $\nu$           & [$m^2/s$]    & kinematische Viskosit"at des
Str"omungsmediums \\
  \vspace{1mm}
  $\rho$          & [$kg/m^3$]   & Dichte des Str"omungsmediums \\
\end{tabular}


\pagestyle{plain}
\renewcommand{\chaptermark}[1]{\markboth{#1}{}}
\renewcommand{\sectionmark}[1]{\markboth{\thesection\ #1}{}}
\lhead[\fancyplain{}{\sl\leftmark}]%
      {\fancyplain{}{\sl\leftmark}}
\rhead[\fancyplain{}{\sl\thepage}]%
      {\fancyplain{}{\sl\thepage}}
\cfoot{}

% Seitennumerierung ab der folgenden Einleitung auf arabische Zahlen setzen
\pagenumbering{arabic}

% Einleitung
\newpage
\chapter{Einleitung}
Ich will hier meinen Text sehen!
Sie f"uhrt in die Problematik ein, skizziert die Motivation und
Zielsetzung sowie das geplante Vorgehen und die angestrebten
Ergebnisse und sollte ca. 1 - 2 Seiten umfassen.

Lorem ipsum dolor sit amet, consectetuer adipiscing elit. Cras elit. Nunc tempor tortor in leo. Nam lectus tortor, pharetra pellentesque, iaculis ut, pretium sed, sem. Nunc congue, sapien id euismod congue, nisl enim mollis sapien, sed suscipit turpis turpis vitae diam. Praesent quam. Cras eleifend. Morbi elementum fermentum tellus. Morbi arcu metus, laoreet molestie, sodales quis, luctus non, eros. Cras ligula. Sed ultrices. Nullam interdum nonummy lectus. Quisque congue hendrerit libero. Donec urna. Vestibulum luctus, massa non pulvinar nonummy, erat ipsum ultricies tortor, a convallis nibh mi non orci. 

Vivamus diam libero, blandit a, malesuada in, egestas accumsan, nunc. Nulla a tellus. Nullam varius. Donec commodo felis in dolor. Cras eleifend, tellus commodo mollis gravida, orci dui iaculis elit, sit amet scelerisque arcu justo non diam. Aenean ipsum lacus, rutrum vel, bibendum vitae, laoreet sed, nisl. Nunc iaculis ante vestibulum odio. Cum sociis natoque penatibus et magnis dis parturient montes, nascetur ridiculus mus. Duis nec sapien. Quisque interdum quam imperdiet eros. Nulla vitae arcu cursus ante pharetra tincidunt. In at mauris. Phasellus pharetra, mi eu accumsan commodo, diam odio consectetuer ligula, a ullamcorper nulla augue et augue. Vivamus diam. Curabitur at lorem. Vestibulum volutpat leo quis metus. Proin metus neque, dapibus a, laoreet quis, ullamcorper eget, magna. 

Nam eu dolor a nisl faucibus suscipit. Nulla interdum sapien id lectus. Curabitur fringilla pulvinar nibh. Aenean porta luctus purus. Cras dictum mauris quis velit. Nullam pharetra pede at risus. Nullam orci sapien, porttitor eu, iaculis et, bibendum ultricies, ipsum. Mauris eget justo. Donec semper auctor tortor. Mauris a ante et magna facilisis mollis. Proin sem turpis, interdum quis, fermentum aliquet, faucibus scelerisque, quam. In mi nibh, facilisis eu, euismod sed, luctus ut, sapien. Etiam ut dui eget libero dapibus elementum. 
 



% Veruchsanordnung bzw. verwendete Software ******************

% Text
\newpage
\chapter{Grundlagen}
Die Grundlage zur Verifizierung ist SunwinT. Um zu verstehen, welche Grundlagen untersucht werden sollen, wird in diesem Kapitel zun"achst eine "Ubersicht "uber die physikalischen Grundlagen und sp"ater eine kurze Herleitung des mathematischen Verfahrens zur L"osung der Differentialgleichungen gegeben.


\section{Str"omungsmechanische Grundlagen}
Um eine Str"omung physikalisch beschreiben zu k"onnen, werden mehrere Erhaltungss"atze angewendet:
\begin{itemize}
	\item Massenerhaltung
	\item Impulsverhaltung (in vektorieller Form)
	\item Energieerhaltung
\end{itemize}

\subsection{Euler-Gleichungen}
Durch Vernachl"assigung der W"arme"ubertragung und Reibung können die Navier-Stokes-Gleichungen zu den Euler-Gleichungen vereinfacht werden.
\begin{gather}
	\frac{\partial\vec{U}}{\partial t} + \nabla \cdot \vec(F_{inv}(U)) = 0 \footnote{Dabei wird hier noch vernachlässigt, dass diese Gleichung nicht immer zwingend 0 sein muss. (Bsp.: Kraftterme oder Method of Manufactured Solutions)}
\end{gather}
$\vec{U}$ beschreibt dabei die konservativen Erhaltungsgr"oßen
\begin{gather}
	\vec{U}=\left(\begin{array}{c} \rho \\ \rho u \\ \rho v \\ \rho w \\ \rho E \end{array}\right) \qquad ,
\end{gather}
wobei $\rho$ f"ur die Dichte, $\vec(v)=\left(\begin{smallmatrix}u, & v, & w\end{smallmatrix}\right)$ f"ur die Geschwindigkeiten in drei Raumdimensionen und E f"ur die totale Energie steht, welche sich aus der inneren Energie e und der kinetischen Energie zusammensetzt
\begin{gather}
	E = e + \frac{1}{2}(u^{2}+v^{2}+w^{2}).
\end{gather}
$\vec(F_{inv}(U))$ bezeichnet in Gl.(2.1) den reibungsfreien Flusstensor $\vec(F_{inv}=(F^{x}_{inv}, F^{y}_{inv}, F^{z}_{inv}, ))$:
\begin{gather}
	\vec{F_{inv}^{x}}=\left(\begin{array}{c} \rho u\\ \rho u^{2}+p\\ \rho uv \\ \rho uw \\ (\rho E+p)u \end{array}\right) \qquad , \vec{F_{i}^{y}}=\left(\begin{array}{c} \rho v\\ \rho uv\\ \rho v^{2}+p \\ \rho vw \\ (\rho E+p)v \end{array}\right) \qquad , \vec{F_{i}^{z}}=\left(\begin{array}{c} \rho w\\ \rho uw\\ \rho vw \\ \rho w^{2}+p \\ (\rho E+p)w \end{array}\right) \qquad  
\end{gather}
Unter Zuhilfenahme der thermischen und kalorischen Zustandsgleichung f"ur ideale Gase schlie"st sich schließlich das Gleichungssystem
\begin{gather}
p = (\gamma -1)(\rho E - \frac{1}{2}(u^{2}+v^{2}+w^{2})).
\end{gather}
Dabei repräsentiert $\gamma$ den adiabaten Exponent. Für Luft gilt $\gamma$=1.4.

\subsection{Navier-Stokes-Gleichungen}
Sollen nun auch Reibung und W"arme"ubertragung ber"ucksichtigt werden, m"uessen die bereits vorgestellten Euler-Gleichungen (Gl. 2.1) um einen reibungsbehafteten Flussterm $F_{vis}$ erweitert werden. Dadurch wird eine Str"omung so allgemein wie m"oglich beschrieben.
\begin{gather}
	\frac{\partial\vec{U}}{\partial t} + \nabla \cdot \vec(F_{inv}(U)) -  \nabla \cdot \vec(F_{vis}(U, \nabla U))= 0
\end{gather}
Wie auch der reibungslose Flussterm, repräsentiert auch der viskose Term $F_{vis}=(F^{x}_{vis}, (F^{y}_{vis}, (F^{z}_{vis})$ alle Raumrichtungen. Dieser berücksichtigt zum einen den Spannungstensor $\tau$
\begin{gather}
	\tau=\begin{pmatrix} 
	\tau_{xx} & \tau_{xy} & \tau_{xz}\\
	\tau_{yx} & \tau_{yy} & \tau_{yz}\\
	\tau_{zx} & \tau_{zy} & \tau_{zz}
	\end{pmatrix}
	\quad,
\end{gather}
für den gilt:
\begin{gather}
	\tau_{ij}=\mu (\frac{\partial u_{i}}{\partial x_{j}} +\frac{\partial u_{j}}{\partial x_{i}} - \frac{2}{3}\frac{\partial u_{k}}{\partial x_{k}}\delta_{ij} )
\end{gather}
\section{Disconstinous Galerkin Verfahren}
In diesem Kapitel wird das in SUNWinT implementierte DG Verfahren skizziert. Da ein räumlich und zeitlich getrennter Ansatz gewählt wurde, wird zunächst auf die räumliche und im Anschluss daran auf die zeitliche Diskretisierung eingegangen.

\subsection{R"aumliche Diskretisierung}
Als Grundlage dient die Euler-Gleichung in differentieller konservativer Form


\newpage
\chapter{Verifizierung}
Beim Modellieren und Simulieren kann nach Roy \cite{roy} prinzipiell in zwei Fehlerquellen unterschieden werden. Es gibt physikalische Modelierungsfehler und mathematische Fehler. Ersteres wird der Sparte der Validierung zugeordnet, die sich unter anderem mit falschen Vereinfachungen besch"aftigt oder dem G"ultikeitsbereich bestimmter Modelle. Dies steht bei der Pr"ufung eines implementierten Verfahrens allerdings an letzter Stelle, da daf"ur die Richtigkeit der Gleichungen und des Codes vorrausgesetzt werden muss.
Die Verifizierung der Gleichungen sowie die Verifizierung des Codes sind beides rein mathematische Verfahren, die lediglich pr"ufen ob die Gleichungen richtig gel"ost werden. Die Validierung hingegen "uberpr"uft, ob die richtigen Gleichungen gel"ost werden\cite{tremblay}.

Nach Trembkay et Al. \cite{tremblay} dient die Verifizierung des Codes dazu nachzuweisen, dass der numerische Code richtig funktioniert (u.a. keine Bugs). Dazu wird der Fehler mit einer bekannten L"osung verglichen und bewertet. Die Verifizierung der Gleichungen hingegen "uberpr"uft die erwartete Genauigkeit bezogen auf bestimmte Anwendungsprobleme. An erster Stelle bei der "Uberpr"ufung von Code sollte immer die Verifizierung des Codes stehen, mit der sich diese Arbeit befasst.


\section{Methoden zur Verifizierung des Codes}
Gerade bei numerischem Code ist auf den Prozess der Verifizierung besonderes Augenmerk zu legen, da die Ausgaben des Codes nicht vom Benutzer als richtig oder flasch unterschieden werden können. Das Ergebnis ist von Parametern wie der Schrittweite der Zeitintegration, dem Netz und dem gew"ahlten Algorithmus abh"angig. Um Code zu verifizieren gibt es verschieden Ans"atze:
\begin{itemize}
	\item \textbf{einfache Tests}\\
	Diese Tests haben den Vorteil, dass sie keine exakten Lösungen brauchen und schnell durchzuf"uhren sind. Sie sind allerdings nicht quantifizierbar und h"angen stark von der F"ahigkeit des Anwenders ab, solche Tests durchf"uhren zu können.
	Ein Beispiel hierf"ur ist eine Strömung um einen Zylinder. Dabei erwartet der Anwender je nach Reynolds-Zahl eine symmetrische Umströmung bzw. das entstehen einer Karmanschen Wirbelstra"se. Erzeugt der geteste Algorithmus andere Ausgaben, muss noch ein Fehler im Code sein.
	F"ur stichhaltige Verifizierungsvorg"ange ist diese Methode allerdings nicht brauchbar.
	\item \textbf{Code-to-Code Verlgeich}\\
	Diese Methode vergleicht nicht syntaktisch jede Zeile des Codes, sondern der vom Code erzeugte Output. Diese Verifzisierungsmethode ist nur dann sinnvoll einsetzbar, wenn beide Codes die selben mathematischen Modelle benutzen und einer der Codes bereits gründlicheren Verifizierungsmethoden unterzogen wurde. Aber auch dann ist ein Code-to-Code Verlgeich nach \cite{bookMMS} mit Vorsicht zu genießen.
	\item \textbf{Quantifizierung des Diskretisierungsfehlers}\\
	Um den Diskretisierungsfehler ermitteln zu können, muss bei dieser Methode eine exakte Lösung vorhanden sein. Nun wird diese exakte Lösung mit der numerischen Lösung bei einem Zeitschritt und einer Diskretesierungsstufe verglichen. Da nur eine Rechnung bzw. Diskretisierung notwendig ist, lässt sich ein solcher Test schnell durchführen. Er verlangt jedoch die subjektive Einschätzung ob der Fehler "klein genug" ist, oder nicht.
	\item \textbf{Konvergenztests}\\
	Zur Durchführung von Konvergenztests wird die selbe Methode wie bei der Quantifizierung des Diskretisierungsfehlers verwendet. Dabei werden die Rechnungen allerdings mit unterschieldichen Zeitschritten oder räumlichen Diskretiserungen durchgeführt. So wird nicht nur die Größe des Fehlers überprüft, sondern auch wie er sich bei feinerern Diskretisierungen verhält. Nach \cite{bookMMS} ist dieser Test die minimale Anforderung an vertrauenswürdige Codeverifizierung.
	\item \textbf{Test der Konvergenzordnung}
	Der Gründlichste Test zur Verifizierung ist der Test der Konvergenzordnung, der im Vergelich zum Konvergenztest auch prüft, ob die theoretische Konvergenzordnung mit der des ausgeführten Codes übereinstimmt. Dazu werden mindestens zwei unterschiedliche Diskretisierungen benötigt, deren Fehler verglichen werden. Die Erfüllung dieses Tests ist das schwierigste und vertrauenswürdigste Kriterium, da der Test auf sehr viele Codingfehler sensibel reagiert (z.B. Transformationen, Implementierung der Randbedingungen etc.)
\end{itemize}

\subsection{Test der Konvergenzordnung}

\section{Generierung der exakten Lösungen}
Wie gezeigt wurde, werden für den Test der Konvergenzordnung analytische Lösungen benötigt. Wie diese erzeugt werden klären die nächsten Abschnitte
\subsection{Methode der exact solutions}
Bei der Methode der exact solutions werden zu Verifizierung nur Gleichungen verwendet, deren L"osung analytisch bestimmt werden kann. Das Vorgehen l"asst sich folgenderma"sen zusammenfassen:
\begin{enumerate}
	\item Wahl der PDEs (f"ur CFD-Anwendungen meist Gl. 2.1 bzw, Gl. 2.6)
	\item Wahl eines Rechengebietes bzw. der Domain
	\item Wahl der Rand-/Anfangswerten
	\item Analytische L"osung der PDEs finden (z.B. Seperation der Variablen, Charakteristikmethode, Transformation)
\end{enumerate} 
Dabei darf nicht vernachl"assigt werden, dass nur f"ur sehr wenige PDE-Systeme eine analytische L"osung existiert. So m"ussen h"aufig relevante Vereinfachungen bei der Wahl der Dimensionen, der Geometrie der zu untersuchenden K"orpern, dem physikalischen Modell oder "ahnlichem gemacht werden. Dies begrenzt die Anwendung auf wenige, vereinfachte Gleichungen.
Als Beispiel hierf"ur f"uhrt Oberkampf et Al. \cite{bookMMS} die Couette-Str"omung an. Es handelt sich dabei um einen der wenigen Str"omungsf"alle, f"ur den eine exakte analytische L"osung existiert. Eine Coutte-Str"omung beschreibt die laminare, station"are Str"omung einer Fl"ussigkeit zwischen zwei parallel zueinander liegenden Platten, wobei sich eine der Platten relativ zur anderen bewegt. Da die viskose Fl"ussigkeit an beiden Platten haftet, stellt sich ein linearer Geschwindigeitsverlauf zwischen den Platten ein. Dadurch muss der diffusive Teil der NS-Gleichungen, ein Term der zweiten Ableitung der Geschwindigkeit null sein. Eine Verifizierung des Codes durch die Methode der exact solution mit der Couette-Str"omung kann floglich nicht "uberpr"ufen, ob der diffusive Term richtig implementiert ist. "Ahnliches gilt auch f"ur die andern analytischl"osbaren F"alle.
Wie das genannte Beispiel zeigt, kann somit nicht garantiert werden, dass die Komplexit"at des implementierten Codes mit all seinen Anwendungsf"allen vollst"andig verifiziert wird.

\subsection{Methode der Manufactured Solutions}
Besonders um die Konvergenzordnung des implementierten Codes zu verifizieren, sind exakte analytische L"osungen von besonderer Bedeutung. Solche exakten L"osungen k"onnen mit der Methode der Manufactured Solutions auch f"ur komplexe, d.h. nicht-lineare, gekoppelte, h"oherdimensionale PDEs gefunden werden, bei denen die traditionellen exakten L"osungen an ihre Grenzen sto"sen. Im traditionellen Ansatz wird versucht eine analytische L"osung durch vorgegebene Rand- bzw. Initialbedingungen zu bestimmen. Die Methode der Manucatured Solutions hingegen nutz zur reinen Verifizierung des Codes aus, dass keine physikalisch sinnvolle L"osung untersucht werden muss. Dies erm"oglicht die Vorgabe einer beliebigen L"osung auf dem Rechengebiet der PDEs. Dies kann erreicht werden, indem die rechte Seite der Gleichung \ref{eq:euler} bzw. \ref{eq:ns} durch einen analytischen Quellterm vorgegeben wird.
Vorgehen nach Roy et Al \cite{roy} um die Konvergenzordnung zu bestimmen
\begin{enumerate}
	\item Festlegung des zu l"osenden PDE-Systems
	\item Wahl der manufactured solutions f"ur jede Zustandsvariable
	\item Anwendung der manufactured solutions auf das PDE-System um analytische Quellterme zu erzeugen
	\item Durchf"uhren der numerischen Rechnungen auf mehrereren unterschiedlich feinen Gittern
	\item Auswerten des globalen Diskretisierungsfehlers in der numerischen L"osung
	\item Abgleich der daraus entstandenen Konvergenzordnung mit der formalen Konvergenzordnung
\end{enumerate}

\subsubsection{Richtlinien f"ur die Wahl der Manufactured solutions}
Um verwertbare Ergebnisse mit dieser Methode zu bekommen, ist es sinnvoll die Richtlinien von Oberkampf und Roy \cite{bookMMS} zu beachten. Demnach muss unteranderem die L"osung zwar nicht physikalisch sein, Naturgesetze, wie keine negativen Temperaturen, m"ussen dennoch eingehalten werden. Eine Verletzung k"onnte bei der Berechnung der Schallgeschwindigkeit zu einem negativen Term unter der Wurzel f"uhren.
Alle Teile des Codes k"onnen nur dann richtig verifiziert werden, wenn kein Term der Manuctured Solutions den anderen um ein Vielfaches "ubersteigt. So wird beispielsweise mit der Wahl einer sehr niedrigen Renoldszahl gew"ahrleistet, dass sowohl der konvektive Teil, wie auch der diffusive Teil der governing equations verifiziert wird. Um dies nachzuweisen sei auf das Vorgehen von Roy at Al. \cite{roy2007verification} verwiesen, die das Verh"altnis der Terme in wichtigen Regionen "uberpr"ufen.
Wichtig bei der Wahl der Manufactured solutions ist auch, dass die Ableitungen, dazu z"ahlen auch mehrfache und gemischte Ableitungen, stetig sind und nicht verschwinden. Aus diesem Grund werden meist trigonometrische oder Exponentialfunktionen verwendet.
Um besonders bei h"oherdimensionalen Rechnungen auf vielfach verfeinerten Gittern Resourcen zu sparen, ist es empfehlenswert nicht eine ganze Periode der trigonometrischen Funktionen auf der Domain abzubilden. F"ur korrekte Ergebnisse reichen breits $\frac{1}{3}$ oder $\frac{1}{5}$.

% Ergebnisse ******************

% Text
\newpage
\chapter{Ergebnisse}
Der Darstellung der Ergebnisse ist ein besonderer Abschnitt zu widmen,
evtl. unterteilt in experimentelle und theoretische Ergebnisse und
Vergleich zwischen Theorie und Messung. Aus der Diskussion der
Ergebnisse sind auch Schlussfolgerungen und Empfehlungen
abzuleiten. Auftretende Diskrepanzen und unplausible Ergebnisse sind
klar herauszustellen, m"ogliche Ursachen sind zu diskutieren.

\section{Verschiedene Sektionen}
\subsection{Verschiedene Untersektionen}



% Zusammenfassung ******************

% Text
\newpage
\chapter{Zusammenfassung}
F"ur den eiligen Leser ist die Vorgehensweise zusammen mit den
wesentlichen Ergebnissen am Schluss in einer ``Zusammenfassung'' klar
herauszustellen. Diese soll ausf"uhrlicher sein als die  ''"Ubersicht'' am Anfang der Arbeit. Auch diese Zusammenfassung soll m"oglichst keine Formeln enthalten.



% Literatur
\bibliographystyle{literaturstil}
\bibliography{literatur}

\clearpage
% Anhang ******************

%Text
\addcontentsline{toc}{chapter}{Anhang}

\newpage
\chapter*{Anhang A}
In diesem Abschnitt soll das Prinzip der Methode der Manufactured Solutions am Beispiel der zweidimensionalen Eulergleichung Gl. \ref{eq:euler} gezeigt werden. Daf"ur werden die von Glasby \cite{glasby2013comparison} vorgestellten Gleichungen als Manufactured Solutions eingesetzt:
\begin{equation}\label{key1}
	\rho = 0,5 (\sin(x^+y^)+1,5)
\end{equation}
\begin{equation}\label{key2}
	u = \sin(x^+y^2)+0,5
\end{equation}
\begin{equation}\label{key3}
	v = 0,1(\cos(x^+y^2)+0,5)
\end{equation}
\begin{equation}\label{key4}
	e = 0,5(\cos(x^2+y^2)+1,5)
\end{equation}
Wie Gl. \ref{zustandsvariablen} entnommen werden kann, braucht SUNWinT konservative Zustandsvaraiblen, weshalb die oben stehenden Gleichungen entsprechend miteinander multipliziert werden m"ussen.
Um nun die entsprechenden Quellterme berechnen zu k"onnen, m"ussen die entsprechenden konservativen Eulergleichungen gel"ost werden. Der "ubersichtlichkeit wegen werden Gl. \ref{eq:euler} hier nochmals anders formuliert:
\begin{equation}\label{eq:firstSource}
	\frac{\partial \rho}{\partial t}+\frac{\partial(\rho u)}{\partial x} + \frac{\partial (\rho v)}{\partial y} = f_{source1}
\end{equation}
\begin{equation}\label{eq:secondSource}
	\frac{\partial \rho u}{\partial t}+\frac{\partial(\rho u^+p)}{\partial x} + \frac{\partial (\rho u v)}{\partial y} = f_{source2}
\end{equation}
\begin{equation}\label{eq:thirdSource}
\frac{\partial \rho v}{\partial t} + \frac{\partial (\rho v u)}{\partial x} + \frac{\partial(\rho v^+p)}{\partial y} = f_{source3}
\end{equation}
\begin{equation}\label{eq:fourthSource}
\frac{\partial \rho e}{\partial t} + \frac{\partial ((\rho e+p)u)}{\partial x} +\frac{\partial ((\rho e+p)v)}{\partial y} = f_{source4}
\end{equation}

In diese Gleichungen k!nne nun unsere manufactured solutions eingesetzt werden um die Quellterme $f_{source1}$-$f_{source4}$ zu bestimmen. Da Gl. \ref{key1} - \ref{key4} nicht von der Zeit abh"angen verschwindet jeweils der erste Term, die darauf folgeden werden nach den "ublichen mathematischen Ableitungsregeln abgeleitet. Da die Ableitungen je nach Ausgangsgleichung und gew"ahlten Manufactured solutions teilweise sehr un"ubersichtilich werden, empfiehlt es sich diese mit einer Software wie Matlab zu erzeugen. 


\end{document}


--PART-BOUNDARY=.19810211828.ZM14332.iag.uni-stuttgart.de--

