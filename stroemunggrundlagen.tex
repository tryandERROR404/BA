\newpage
\chapter{Str"omungsmechanische Grundlagen}
Die Grundlage zur Verifizierung ist SunwinT. Um zu verstehen, welche Grundlagen untersucht werden sollen, wird in diesem Kapitel zun"achst eine "Ubersicht "uber die physikalischen Grundlagen gegeben

Um eine Str"omung physikalisch beschreiben zu k"onnen, werden mehrere Erhaltungss"atze angewendet:
\begin{itemize}
	\item Massenerhaltung: $\frac{\partial \rho}{\partial t} + \nabla \cdot (\rho u) = 0$
	\item Impulsverhaltung (in vektorieller Form): $\frac{\partial (\rho \vec{v})}{\partial t} + \nabla \cdot (\rho \vec v \vec v ) + \nabla \cdot p - \nabla \cdot \bar{\bar{\tau}} = 0$
	\item Energieerhaltung: $\frac{\partial (\rho E)}{\partial t} + \nabla \cdot (\rho \vec{v} E) + \nabla \cdot (p \vec{v}) - \nabla \bar{\bar{\tau}} \vec{v} + \nabla \cdot \vec{q} = 0$ 
\end{itemize}

\section{Euler-Gleichungen}
Durch Vernachl"assigung der W"arme"ubertragung und Reibung können die Navier-Stokes-Gleichungen zu den instationären inkompressiblen Euler-Gleichungen vereinfacht werden. In differentieller konservativer Form lauten sie:
\begin{equation}\label{eq:euler}
	\frac{\partial\vec{U}}{\partial t} + \nabla \cdot \vec F_{inv}(U) = 0 \footnote{Dabei wird hier noch vernachl"assigt, dass diese Gleichung nicht immer zwingend 0 sein muss. (Bsp.: Kraftterme oder Method of Manufactured Solutions)}	
\end{equation}
$\vec{U}$ beschreibt dabei die konservativen Erhaltungsgr"oßen
\begin{gather}\label{zustandsvariablen}
	\vec{U}=\left(\begin{array}{c} \rho \\ \rho u \\ \rho v \\ \rho w \\ \rho E \end{array}\right) \qquad ,
\end{gather}
wobei $\rho$ f"ur die Dichte, $\vec v=\left(\begin{smallmatrix}u, & v, & w\end{smallmatrix}\right)^{T}$ f"ur die Geschwindigkeiten in drei Raumdimensionen und E f"ur die totale Energie steht, welche sich aus der inneren Energie e und der kinetischen Energie zusammensetzt
\begin{gather}
	E = e + \frac{1}{2}(u^{2}+v^{2}+w^{2}).
\end{gather}
$\vec F_{inv}(U)$ bezeichnet in \ref{eq:euler} den reibungsfreien Flusstensor $\vec F_{inv}=(\vec F^{x}_{inv}, \vec F^{y}_{inv}, \vec{F^{z}_{inv}})$:
\begin{gather}
	\vec{F_{inv}^{x}}=\left(\begin{array}{c} \rho u\\ \rho u^{2}+p\\ \rho uv \\ \rho uw \\ (\rho E+p)u \end{array}\right) \qquad , \vec{F_{i}^{y}}=\left(\begin{array}{c} \rho v\\ \rho uv\\ \rho v^{2}+p \\ \rho vw \\ (\rho E+p)v \end{array}\right) \qquad , \vec{F_{i}^{z}}=\left(\begin{array}{c} \rho w\\ \rho uw\\ \rho vw \\ \rho w^{2}+p \\ (\rho E+p)w \end{array}\right) \qquad  
\end{gather}
Unter Zuhilfenahme der thermischen und kalorischen Zustandsgleichung f"ur ideale Gase schlie"st sich schließlich das Gleichungssystem
\begin{gather}
p = (\gamma -1)(\rho E - \frac{1}{2}(u^{2}+v^{2}+w^{2})).
\end{gather}
Dabei repr"asentiert $\gamma$ den adiabaten Exponent. F"ur Luft gilt $\gamma$=1.4.




\section{Navier-Stokes-Gleichungen}
Sollen nun auch Reibung und W"arme"ubertragung ber"ucksichtigt werden, m"uessen die bereits vorgestellten Euler-Gleichungen (\ref{eq:euler}) um einen reibungsbehafteten Flussterm $\vec F_{vis}$ erweitert werden. Dadurch wird eine Str"omung so allgemein wie m"oglich beschrieben.
\begin{gather}\label{eq:ns}
	\frac{\partial\vec{U}}{\partial t} + \nabla \cdot \vec F_{inv}(U) -  \nabla \cdot \vec F_{vis}(U, \nabla U)= 0
\end{gather}
Wie auch der reibungslose Flussterm, repr"asentiert auch der viskose Term $\vec F_{vis}=(\vec F^{x}_{vis}, \vec F^{y}_{vis}, \vec F^{z}_{vis})$ alle Raumrichtungen. Dieser ber"ucksichtigt einerseits den Spannungstensor $\bar{\bar{\tau}}$
\begin{gather}
	\tau=\begin{pmatrix} 
	\tau_{xx} & \tau_{xy} & \tau_{xz}\\
	\tau_{yx} & \tau_{yy} & \tau_{yz}\\
	\tau_{zx} & \tau_{zy} & \tau_{zz}
	\end{pmatrix}
	\quad,
\end{gather}
f"ur den gilt:
\begin{gather}
	\tau_{ij}=\mu \left(\frac{\partial u_{i}}{\partial x_{j}} +\frac{\partial u_{j}}{\partial x_{i}} - \frac{2}{3}\frac{\partial u_{k}}{\partial x_{k}}\delta_{ij} \right) .
\end{gather}
Dabei repr"asentiert $\delta_{ij}$ das Kronekterdelta mit
\begin{gather}
	\delta_{ij}=\begin{cases} 
	1 & \text{f"ur j=i}\\
	0 & \text{sonst}\\
	\end{cases}.
\end{gather}
F"ur die dynamische Viskosit"at $\mu$ wird Sutherlands Law angewedent, das die Abh"anigkeit der Viskosit"at mit der Temperatur beschreibt
\begin{gather}
	\mu = \mu_{0} \left(\frac{T}{T_{0}}\right)^\frac{3}{2} \frac{T_{0}+S}{T+S}
\end{gather}
mit
\begin{gather*}
	\mu_{0}=1,716\cdot10^{-5} \frac{kg}{ms}, \\
	T_{0}=273,15 K  \  \text{und} \\
	S = 110,55K .
\end{gather*}
Andererseits wird bei den NS-Gleichungen in der Zeile der Energieerhaltung auch die W"arme"ubertragung ber"ucksichtigt. Der daf"ur notwendige W"armestromdichtevektor $\vec{q}$ wird durch das Fouriesche Gesetzt erzeugt:
\begin{gather}
	\vec{q} = -\lambda \frac{\partial T}{\partial \vec{x}},
\end{gather}
wobei $\lambda$ die temperaturabh"anige Stoffgröße der W"armeleitf"ahigkeit repr"asentiert und sich wiederum aus der dynamischen Viskosit"at $\mu$, der W"armeleitf"ahigkeit bei konstantem Druck $c_{p}$ so wie der Prandtlzahl Pr berechnen l"asst:
\begin{gather}
	\lambda = \frac{\mu c_{p}}{Pr}
\end{gather}
Die Prandtlzahl wird dabei auf $Pr = 0,72$ festgesetzt, was f"ur Luft zwischen 200 K und 600 K als ann"ahernd konstant angenommen werden kann.
so erh"alt man schließlich die viskosen Flussterme zu
\begin{gather*}
	\vec{F_{vis}^{x}}=\left(\begin{array}{c} 0\\ \tau_{xx}\\ \tau_{xy} \\ \tau_{xz} \\ u\tau_{xx}+v\tau_{xy}+w\tau_{xz}+q_{x} \end{array}\right), \qquad  \vec{F_{vis}^{y}}=\left(\begin{array}{c} 0\\ \tau_{yx}\\ \tau_{yy} \\ \tau_{yz} \\ u\tau_{yx}+v\tau_{yy}+w\tau_{yz}+q_{y} \end{array}\right),  
\end{gather*}
\begin{gather}
	\vec{F_{vis}^{z}}=\left(\begin{array}{c} 0\\ \tau_{zx}\\ \tau_{zy} \\ \tau_{zz} \\ u\tau_{zx}+v\tau_{zy}+w\tau_{zz}+q_{z} \end{array}\right) \qquad 
\end{gather}

