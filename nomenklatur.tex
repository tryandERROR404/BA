\newpage
\chapter*{Nomenklatur}
\begin{tabular}{lll}
  \vspace{1mm}
  $a_i$            & [-]            & Polynomkoeffizient \\
  \vspace{1mm}
  $c_f$            & [-]            & Reibungswiderstandsbeiwert der
turbulent umstr"omten ebenen Platte \\
  \vspace{1mm}
  $c_{w_V}$        & [-]            & volumenbezogener
Widerstandsbeiwert \\
  \vspace{1mm}
  $D$              & [m]             & Durchmesser \\
  \vspace{1mm}
  $L$              & [m]            & K"orperl"ange \\
  \vspace{1mm}
  $n_{krit.}$      & [-]            & kritischer Anfachungsfaktor
(relevant f"ur die Umschlagsberechnung) \\
  \vspace{1mm}
  $Re_L$           & [-]            & l"angenbezogene Reynoldszahl \\
  \vspace{1mm}
  $Re_V$           & [-]            & volumenbezogene Reynoldszahl \\
  \vspace{1mm}
  $r$              & [m]            & Radius \\
  \vspace{1mm}
  $U_{\infty}$     & [$m/s$]        & Anstr"omgeschwindigkeit \\
  \vspace{1mm}
  $V$              & [$m^3$]        & Volumen \\     
   \vspace{1mm}
  $W$              & [$N$]          & Widerstand \\      
  \vspace{1mm}
  $x, y, z$        & [m]            & kartesische Koordinaten \\


  \vspace{1mm}
  $\alpha$        & [$^{\circ}$] & Anstellwinkel \\
  \vspace{1mm}
  $\nu$           & [$m^2/s$]    & kinematische Viskosit"at des
Str"omungsmediums \\
  \vspace{1mm}
  $\rho$          & [$kg/m^3$]   & Dichte des Str"omungsmediums \\
\end{tabular}
